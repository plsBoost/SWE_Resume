%-------------------------
% Resume in Latex
% Author : Jake Gutierrez
% Based off of: https://github.com/sb2nov/resume
% License : MIT
%------------------------

\documentclass[letterpaper,11pt]{article}

\usepackage{latexsym}
\usepackage[empty]{fullpage}
\usepackage{titlesec}
\usepackage{marvosym}
\usepackage[usenames,dvipsnames]{color}
\usepackage{verbatim}
\usepackage{enumitem}
\usepackage[hidelinks]{hyperref}
\usepackage{fancyhdr}
\usepackage[english]{babel}
\usepackage{tabularx}
\input{glyphtounicode}


%----------FONT OPTIONS----------
% sans-serif
% \usepackage[sfdefault]{FiraSans}
% \usepackage[sfdefault]{roboto}
% \usepackage[sfdefault]{noto-sans}
% \usepackage[default]{sourcesanspro}

% serif
% \usepackage{CormorantGaramond}
\usepackage{charter}


\pagestyle{fancy}
\fancyhf{} % clear all header and footer fields
\fancyfoot{}
\renewcommand{\headrulewidth}{0pt}
\renewcommand{\footrulewidth}{0pt}

% Adjust margins
\addtolength{\oddsidemargin}{-0.5in}
\addtolength{\evensidemargin}{-0.5in}
\addtolength{\textwidth}{1in}
\addtolength{\topmargin}{-.5in}
\addtolength{\textheight}{1.0in}

\urlstyle{same}

\raggedbottom
\raggedright
\setlength{\tabcolsep}{0in}

% Sections formatting
\titleformat{\section}{
  \vspace{-4pt}\scshape\raggedright\large
}{}{0em}{}[\color{black}\titlerule \vspace{-5pt}]

% Ensure that generate pdf is machine readable/ATS parsable
\pdfgentounicode=1

%-------------------------
% Custom commands
\newcommand{\resumeItem}[1]{
  \item\small{
    {#1 \vspace{-2pt}}
  }
}

\newcommand{\resumeSubheading}[4]{
  \vspace{-2pt}\item
    \begin{tabular*}{0.97\textwidth}[t]{l@{\extracolsep{\fill}}r}
      \textbf{#1} & #2 \\
      \textit{\small#3} & \textit{\small #4} \\
    \end{tabular*}\vspace{-7pt}
}

\newcommand{\resumeSubSubheading}[2]{
    \item
    \begin{tabular*}{0.97\textwidth}{l@{\extracolsep{\fill}}r}
      \textit{\small#1} & \textit{\small #2} \\
    \end{tabular*}\vspace{-7pt}
}

\newcommand{\resumeProjectHeading}[2]{
    \item
    \begin{tabular*}{0.97\textwidth}{l@{\extracolsep{\fill}}r}
      \small#1 & #2 \\
    \end{tabular*}\vspace{-7pt}
}

\newcommand{\resumeSubItem}[1]{\resumeItem{#1}\vspace{-4pt}}

\renewcommand\labelitemii{$\vcenter{\hbox{\tiny$\bullet$}}$}

\newcommand{\resumeSubHeadingListStart}{\begin{itemize}[leftmargin=0.15in, label={}]}
\newcommand{\resumeSubHeadingListEnd}{\end{itemize}}
\newcommand{\resumeItemListStart}{\begin{itemize}}
\newcommand{\resumeItemListEnd}{\end{itemize}\vspace{-5pt}}

%-------------------------------------------
%%%%%%  RESUME STARTS HERE  %%%%%%%%%%%%%%%%%%%%%%%%%%%%

%Compiler wants this command v
\setlength{\footskip}{5pt}

\begin{document}

%----------HEADING----------
% \begin{tabular*}{\textwidth}{l@{\extracolsep{\fill}}r}
%   \textbf{\href{http://sourabhbajaj.com/}{\Large Sourabh Bajaj}} & Email : \href{mailto:sourabh@sourabhbajaj.com}{sourabh@sourabhbajaj.com}\\
%   \href{http://sourabhbajaj.com/}{http://www.sourabhbajaj.com} & Mobile : +1-123-456-7890 \\
% \end{tabular*}

\begin{center}
  \textbf{\Huge \scshape Adnan Aman} \\ \vspace{1pt}
  \small 949-247-9312 $|$ \href{mailto:adnan_aman@berkeley.edu}{ \color{blue} \underline{adnan\_aman@berkeley.edu}} $|$
  \href{https://linkedin.com/in/adnan-aman}{\color{blue} \underline{linkedin.com/in/adnan-aman}} $|$
  \href{https://github.com/plsBoost}{ \color{blue} \underline{github.com/plsBoost}}
\end{center}

\vspace*{-3mm}
%-----------EDUCATION-----------
\section{Education}
\resumeSubHeadingListStart

\vspace{-1pt}\item
\begin{tabular*}{0.97\textwidth}[t]{l@{\extracolsep{\fill}}r}
  \textbf{University of California, Berkeley} & \textbf{Class of 2025} \\
  \textit{\small Bachelor of Arts in Computer Science} &  GPA: 3.6/4.0 \\
\end{tabular*}\vspace{-7pt}

\vspace{2pt}\item
\begin{tabular*}{0.97\textwidth}[t]{l@{\extracolsep{\fill}}r}
  \textbf{Relevant Coursework:} \\
\end{tabular*}\vspace{-20pt}




% \begin{tabular*}
%     \small{
%      \textbf{Languages}{: Java, Python, C, SQL, HTML/CSS} \\
%     %  \textbf{Frameworks}{: React, Node.js, Flask, JUnit, WordPress, Material-UI, FastAPI} \\
%      \textbf{Developer Tools}{: Git, Vim, VS Code, Linux, IntelliJ IDEA, Android Studio, Visual Studio, Eclipse} \\
%      % \textbf{Libraries}{: pandas, NumPy}
%     }
%  \end{tabular*}

% \begin{tabular*}{0.97\textwidth}[t]{l@{\extracolsep{\fill}}r}
%   \textbf{UC Berkeley} & \textbf{Class of 2024} \\
%   \textit{\small Bachelor of Arts in Computer Science} &  GPA: 3.85/4.0 \\
% \end{tabular*}\vspace{7pt}

\hfill


\small {
  Operating Systems, Networks, Data Structures, Efficient Algorithms and Intractable Problems, Computer Architecture, Introduction to Database Systems, Computer Security, Discrete Math and Probability, Optimization Models in Engineering, Machine Learning, and Probability for Data Science

}

\resumeSubHeadingListEnd

%-----------EXPERIENCE-----------
\vspace{-4 mm}
\section{Experience}
\vspace{0pt}
\resumeSubHeadingListStart

\resumeSubheading
{Microsoft}{May 2024 -- Present}
{Software Engineer Intern}{Redmond, WA}
\resumeItemListStart
\resumeItem{Implemented distributed tracing for the Azure Machine Learning inferencing team, enhancing observability for over \textbf{2 trillion} monthly scoring requests using Go, Docker, YAML, and Python}
\resumeItem{Configured and deployed OTel tracing agent, facilitating the exporting of traces across multiple distributed systems}
\resumeItem{Integrated tracing capabilities into HTTP listeners using Envoy configurations, enhancing service observability}
\resumeItem{Instrumented spans for distributed tracing within Go applications using OpenTelemetry SDK, enabling traceability}
\resumeItem{Configured and onboarded a visualization tool for monitoring traces and connected services, improving debugging efficiency}
\resumeItem{Collaborated with cross-functional teams to optimize and scale distributed tracing solutions through performance testing, resulting in more reliable and scalable systems}
\resumeItemListEnd

\resumeSubheading
{University of California, Berkeley}{June 2023 -- August 2023}
{Academic Intern}{Berkeley, CA}
\resumeItemListStart
\resumeItem{Worked as a lab assistant for UC Berkeley’s Data Structures course with roughly $\sim$1600 students}
\resumeItem{Assisted students with projects and labs in Java, helping to increase their coding and debugging skills}
\resumeItem{Worked in office hours to support students with homework and conceptual misunderstandings}
\resumeItem{Guided students to implement and experiment with fundamental algorithms and data structures through various projects and labs}
\resumeItemListEnd

%\resumeSubheading
%{CodePath}{August 2021 -- January 2022}
%{Android Software Engineer}{Irvine, CA}
%\resumeItemListStart
%\resumeItem{Employed MVC patterns in 3 major projects, leading to a modular codebase, which improved maintainability and allowed a responsive user experience for thousands of active users}
%\resumeItem{Integrated RESTful APIs using CodePath's AsyncHttpLibrary in 4 applications, facilitating real-time data fetch and display, leading to a \textbf{25\%} improvement in data load times}
%\resumeItem{Enhanced app security by pioneering advanced user authentication techniques, which reduced security breaches by \textbf{50\%} and streamlined user onboarding}
%\resumeItemListEnd

\resumeSubHeadingListEnd
\vspace*{-3.5mm}
%-----------PROJECTS-----------
\section{Projects}
\resumeSubHeadingListStart

\resumeProjectHeading
{\textbf{YelpCamp} $|$ \emph{Node.js, Express.js, MongoDB, Bootstrap}}{December 2023 -- Present}
\resumeItemListStart
\resumeItem{Developed a web application for campsite reviews, using Node.js, Express.js, and MongoDB, focusing on user-generated content, security, and data integrity}
\resumeItem{Implemented user authentication, admin roles, and a review system in YelpCamp, to enhance application security and user interaction}
\resumeItem{Integrated Google Maps API for interactive campsite location features and deployed Google Ads for revenue generation}
%\resumeItem{Employed MVC architecture for application design, ensuring scalability and maintenance efficiency in the codebase}
\resumeItemListEnd

%\resumeProjectHeading
%{\textbf{Gaussian Discriminant Analysis} $|$ \emph{Python, NumPy, Scikit-learn, Matplotlib}}{February 2024}
%\resumeItemListStart
%\resumeItem{Developed a Gaussian discriminant analysis model to classify digit images and identify spam emails}
%\resumeItem{Applied feature extraction and dimensionality reduction techniques on the MNIST dataset for efficient digit classification}
%\resumeItem{Implemented Linear Discriminant Analysis (LDA) and Quadratic Discriminant Analysis (QDA) for model training, resulting in low test error rate}
%\resumeItem{Employed cross-validation and hyperparameter tuning to optimize model parameters, resulting in improved validation accuracy for both digit and spam classification}
%\resumeItem{Visualized data distributions and model decision boundaries using Matplotlib, to understand the model's performance}
%\resumeItemListEnd


% \resumeProjectHeading
% {\textbf{Secure File Sharing System} $|$ \emph{Go, Cryptography}}{January 2024 -- March 2024}
% \resumeItemListStart
% \resumeItem{Developed a secure file sharing system in Go, allowing users to create, edit, and share files}
% \resumeItem{Used cryptographic libraries for encryption, decryption, and secure invitation links, to ensure data confidentiality and integrity even in the presence of attackers}
% \resumeItem{Focused on resilience and security through atomic operations, asymmetric encryption (RSA), symmetric encryption (AES-CTR), hash-based MACs, signatures, and certificates}
% \resumeItem{Used remote databases for persistent, secure data storage, UUIDs for data management, and PKI for authentication.}
% \resumeItemListEnd

% \vspace{-4mm}
\resumeProjectHeading
{\textbf{RookieDB: Resilient Database Recovery System} $|$ \emph{Java, ARIES Algorithm}}{January 2023 -- May 2023}
\resumeItemListStart
\resumeItem{Designed a database recovery system using Java and the ARIES algorithm}
\resumeItem{Used logging and checkpoints for system recovery in case of system failures}
\resumeItem{Optimized I/O operations utilizing concurrency and query optimization, which led to a 30\% reduction in data retrieval latency}
\resumeItemListEnd

\resumeProjectHeading
{\textbf{CS61KaChow: Optimized 2D Convolutions} $|$ \emph{C, SIMD, OpenMP, Open MPI}}{April 2023 -- May 2023}
\resumeItemListStart
\resumeItem{Optimized 2D convolutions utilizing SIMD vector instructions, achieving a \textbf{8.05x} speedup and significantly improving image processing times}
\resumeItem{Enhanced task parallelism using OpenMP, resulting in efficient multi-threaded operations and reduced processing overhead}
\resumeItem{Coordinated parallel processing tasks utilizing Open MPI's manager-worker architecture, leading to a \textbf{5.30x} speedup in convolution operations across large datasets}
\resumeItemListEnd

\resumeSubHeadingListEnd

\vspace*{-3mm}
%-----------PROGRAMMING SKILLS-----------
\section{Technical Skills}
\begin{itemize}[leftmargin=0.15in, label={}]
  \small{\item{
        \textbf{Languages}{: Java, Python, C, Golang, JavaScript, HTML, CSS, SQL, MQL} \\
        \textbf{Frameworks and Libraries}{: React, Node.js, Express.js, Bootstrap, Android (MVC), JUnit, OpenTelemetry} \\
        \textbf{Developer Tools}{: Git, Vim, Linux, MongoDB, LaTeX, Docker, Bash} \\
        }}
        % \textbf{Developer Tools}{: Git, Vim, Linux, MongoDB, LaTeX, Android Studio, Logisim} \\
        % }}
\end{itemize}

%-------------------------------------------
\end{document}
